%!TEX root = ../PatilM-[RnD-MT]Report.tex

\chapter{Approach}
The conclusions drawn in the previous chapter, allow us the liberty of focusing specifically on the two most promising qualitative calculi $QTC_B$ and $ARGD$, for the purpose of developing an algorithm that centers around the utilization of either one or both of these calculi for the purpose of qualitatively representing physical space in our application of mobile robot navigation.

\paragraph{} From the above analysis of the state of the art it is clear that the advantages of qualitative representations lie in the dealing with imprecise raw data, but it should be noted that these representations aren't a replacement for quantitative representation instead they are simply an alternative that can be harnessed to deal with scenarios where the need for crisp, precise data cannot be justified such as traversal in a fairly empty open spaces(halls, corridors etc.). Precise quantitative representations are still necessary when dealing with scenarios such as passing through a door.
\section{Proposed approach}


\section{Implementation details}
